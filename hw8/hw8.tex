\documentclass[12pt,a4paper]{article}


\usepackage{amsmath}
\usepackage{amssymb}
\usepackage{amsthm}

\newtheorem{thm}{Theorem}

\usepackage{algorithm,algorithmicx,algpseudocode}
\usepackage{hyperref}
\usepackage{listings,color}

\title{APKC - Homework 8}
\author{Sebastien Duc}
\date{\today}

\newcommand {\zpz}[1]{\mathbb{Z}/#1\mathbb{Z}}
\newcommand {\polyf}{\mathbb{Z}[x]}
\newcommand {\sign}{\mathrm{sign}}
\newcommand {\thus}{\Rightarrow\:}
\newcommand {\nequiv}{\not\equiv}
\newcommand {\fact}{\mathrm{fact}}
\newcommand {\ord}{\mathrm{ord}}
\newcommand {\lcm}{\mathrm{lcm}}

\definecolor{dkgreen}{rgb}{0,0.6,0}
\definecolor{gray}{rgb}{0.5,0.5,0.5}
\definecolor{mauve}{rgb}{0.58,0,0.82}

\usepackage{xcolor}
\usepackage{caption}
%\DeclareCaptionFont{white}{\color{white}}
%\DeclareCaptionFormat{listing}{%
%  \parbox{\textwidth}{\colorbox{gray}{\parbox{\textwidth}{#1#2#3}}\vskip-4pt}}
%  \captionsetup[lstlisting]{format=listing,labelfont=white,textfont=white}

\lstset{ %
    language=Python,                % the language of the code
    basicstyle=\footnotesize,       % the size of the fonts that are used for the code
    backgroundcolor=\color{white},      % choose the background color. You must add \usepackage{color}
    showspaces=false,               % show spaces adding particular underscores
    showstringspaces=false,         % underline spaces within strings
    showtabs=false,                 % show tabs within strings adding particular underscores
    frame=single,                   % adds a frame around the code
    rulecolor=\color{black},        % if not set, the frame-color may be changed on line-breaks within not-black text (e.g. comments (green here))
    tabsize=2,                      % sets default tabsize to 2 spaces
    captionpos=b,                   % sets the caption-position to bottom
    breaklines=true,                % sets automatic line breaking
    breakatwhitespace=false,        % sets if automatic breaks should only happen at whitespace
    title=\lstname,                   % show the filename of files included with \lstinputlisting;
    % also try caption instead of title
    keywordstyle=\color{blue},          % keyword style
    commentstyle=\color{dkgreen},       % comment style
    stringstyle=\color{mauve},         % string literal style
    escapeinside={\%*}{*)},            % if you want to add LaTeX within your code
    morekeywords={*,...},              % if you want to add more keywords to the set
    deletekeywords={...}              % if you want to delete keywords from the given language
    xleftmargin=\fboxsep,xrightmargin=-\fboxsep
}

\begin{document}
\maketitle

\section*{8.1}

\subsection*{(a)}
Let us first consider the factorization of $f$ into monic distinct irreducible polynomial $p_i$:
\[
    f(X) = c \prod_{i=1}^{k}{p_i(X)^{e_i}} \quad\mbox{,where } 0<e_i\leq n, \forall\: 1 \leq i \leq k \mbox{, and $c\in F_q$}.
\]
We know that this factorization is unique.
Now consider the following polynomials
\[
    f_i(X) = \prod_{1 \leq j \leq k:e_j = i}p_j(X)
\]
for $i = 1$ to $n$ (because $1 \leq e_i \leq n$).
Then we get that $f$ can be uniquely decomposed as follows
\[
    f(X) = c f_1(X)f_2(X)^2\dots f_n(X)^n
\]
Furthermore, when $i\neq j$ $\gcd(f_i(X),f_j(X)) = 1$ because they are both products of monic irreducible primes and
with none of them in both $f_i(X)$ and $f_j(X)$. Finally they are square-free because all the $p_j$'s are square-free and 
the only divisors of the $f_i$'s are the $p_j$'s. Hence the result.


\subsection*{(b)}
We use Yun's algorithm~\cite{wiki} to find the decomposition.
\begin{algorithm}
    \caption{Yun's Algorithm}
    \begin{tabular}{l l}
        \textbf{Input:} & $ f \in \mathbb{F}_q[X] $ of degree $n$\\
        \textbf{Output:} & $ (f_1,\dots,f_n) $ all monic, square free and pairwise relatively prime\\
    \end{tabular}
    \begin{algorithmic}[1]
        \State $f_0 \gets \gcd(f,f')$
        \State $b \gets f/f_0$
        \State $c \gets f'/f_0$
        \State $d \gets c - b'$
        \State $i \gets 1$
        \Repeat
            \State $f_i \gets \gcd(b_i,d_i)$
            \State $b \gets b/f_i$
            \State $c \gets d/f_i$
            \State $i \gets i+1$
            \State $d \gets c - b'$
        \Until{$b=1$}
    \end{algorithmic}
\end{algorithm}
\subsection*{(c)}
In the implementation, only the polynomial is needed as input because the field in which it was defined can be recovered.
Function \texttt{test\_factor\_random} is provided to test on a random polynomial of a given degree in a finite field. 

\belowcaptionskip=-10pt
\begin{lstlisting}[label=fact,caption=Yun's Algorithm]
def is_monic(f):
    return f.coefficients()[-1] == 1

def factor_poly(f):
    return yun_fact(f)

def yun_fact(f):
    f *= f.coefficients()[-1]**(-1)
    out = []
    b = f
    d = f.derivative()
    while not b.is_one():
        a = b.gcd(d)
        assert is_monic(a)
        out.append(a)
        b = b // a
        c = d // a
        d = c - b.derivative()
    return out[1:] 

# tests factor poly on a random poly in F_p^k[X] with a poly of degree n
def test_factor_random(p,k,n):
    Fq.<X> = GF(p**k,name='a')[['X']]
    f = Fq.random_element(n+1).polynomial()
    fact = factor_poly(f)
    print "obtained factorization:" , fact
    
    test_f = f.coefficients()[-1]
    for a,i in zip(fact,range(1,len(fact)+1)):
        assert a.is_squarefree()
        test_f *= a**i
    print test_f , "==" , f
    assert test_f == f
\end{lstlisting}

\subsection*{(d)}
The following two functions are implemented for the purpose of this exercise.
\begin{lstlisting}[caption=Test Factorization]
# (d) part of the exercise
# test polynomial given
def test_factor():
    Fq.<X> = GF(5)[['X']]
    f = 3*X^10 + 4*X^7 + 4*X^5 + X^4 + X^3 + 2*X^2 + X + 4
    print "f has factorization" , factor_poly(f.polynomial())

# (d) part of the exercise
# test with sciper generated polynomial
def test_poly_from_sciper(sciper):
    F.<X> = GF(11)[['X']]
    poly = 1
    for i in range(10):
        if i in sciper.digits():
            poly *= (X - i)**i
    print "f (from sciper) has factorization" , factor_poly(poly.polynomial())
\end{lstlisting}

\section*{8.2}
Let us define the following
\[
    \begin{array}{l l}
        S_g &= \{ s \in \mathbb{F}_q \mid \gcd(f(X),g(X)-s) \neq 1 \} \\
        h_g(X) &= \prod_{s\in S_g} (X - s)\\
    \end{array}
\]
Then if we can compute $h_g$, we just need to find the roots of this polynomial.
Let us see now how to find $h_g$. First consider the following theorem.
\begin{thm}
    Polynomial $h_g(X) = \prod_{s\in S_g} (X - s)$ is the minimal polynomial for $g(X)$. That is, $h_g(X)$ is the polynomial with the smallest degree
    such that \[h_g(g(X)) \equiv 0 \pmod{f(X)}.\]
\end{thm}
\begin{proof}
    Pick an arbitrary factor $f_i(X)$ of $f(X)$, it must be true that $f_i(X) \mid \gcd(f(X),g(X) - s)$ for some $s \in S_g$.
    Therefore $f_i(X) \mid (g(X) - s)$ for some $s\in S_g$. Saying that for an arbitrary factor of $f(X)$ leads us to
    \[
        f(X) \mid \prod_{s\in S_g}(g(X) - s)  = h_g(g(X))
    \]
    Hence $h_g(g(X)) \equiv 0 \pmod{f(X)}$.

    Now we need to show that the degree of $h_g$ is minimal.
    Suppose by contradiction that there exist a polynomial $h$ with smaller degree such that $h(g(X)) \equiv 0 \pmod{f(X)}$.
    Then there must exist an $s \in S_g$ such that $h(X) = q(X) \cdot (X - s) + r$, where $r$ is nonzero element in $\mathbb{F}_q$.
    We know that one factor $f_i$ of $f$ divides $g(X) - s$. But since $h(g(X)) \equiv 0 \pmod{f(X)}$, $f_i$ divides $h(g(X))$.
    Furthermore, $h(g(X)) = q(g(X)) \cdot (g(X) - s) + r$. This implies that $f_i(X) \mid r$ (since $f_i(X)\mid h(g(X))$ and $f_i(X) \mid (X - s)$).
    But $r$ is in $\mathbb{F}_q$, which is a contradiction. Hence the minimality of $h_g$.
\end{proof}

To find the minimal polynomial for $g(X)$, we proceed as follows: we iteratively compute $g(X)^k \bmod f(X)$ starting with $k = 1$. To compute 
$g(X)^{k+1} \bmod f(X)$ we just multiply $g(X)^k \bmod f(X)$ with $g(X) \bmod f(X)$. At each step $k$ we solve the following $n$ by $k$ system, $n$ being the degree of $f$
\[
    h_0 + h_1 g(X) + h_2 g(X)^2 + \dots + h_{k-1}g(X)^{k-1} + g(X)^k \equiv 0 \pmod{f(X)}
\]
until the solution is nontrivial. The nontrivial solution is found at step $k=r$ ($r$ is the number of irreducible factors of $f$).
When the solution $(h_0,h_1,\dots,h_{r-1})$ is found, the polynomial $h_g$ is simply
\[
    h_g(X) = h_0 + h_1 X + \dots + h_{r-1} X^{r-1} + X^r
\]
and we just have to find the roots of this polynomial to factor $f$.


\begin{thebibliography}{1}
    \bibitem{wiki}
        Wikipedia, \emph{Square-free polynonmial}\\
        \url{http://en.wikipedia.org/wiki/Square-free\_polynomial}
\end{thebibliography}

\end{document}
