\documentclass[12pt,a4paper]{article}



%----------------------------------------------------------------------------------------
%	PACKAGES AND OTHER DOCUMENT CONFIGURATIONS
%----------------------------------------------------------------------------------------

\usepackage{fancyhdr} % Required for custom headers
\usepackage{lastpage} % Required to determine the last page for the footer
\usepackage{extramarks} % Required for headers and footers
\usepackage[usenames,dvipsnames]{color} % Required for custom colors
\usepackage{graphicx} % Required to insert images
\usepackage{listings} % Required for insertion of code
\usepackage{courier} % Required for the courier font

% Margins
\topmargin=-0.45in
\evensidemargin=0in
\oddsidemargin=0in
\textwidth=6.5in
\textheight=9.0in
\headsep=0.25in

\linespread{1.1} % Line spacing

% Set up the header and footer
\pagestyle{fancy}
\lhead{\hmwkAuthorName} % Top left header
\chead{\hmwkClass } % Top center head
\rhead{\hmwkTitle\firstxmark} % Top right header
\lfoot{\lastxmark} % Bottom left footer
\cfoot{} % Bottom center footer
\rfoot{Page\ \thepage\ of\ \protect\pageref{LastPage}} % Bottom right footer
\renewcommand\headrulewidth{0.4pt} % Size of the header rule
\renewcommand\footrulewidth{0.4pt} % Size of the footer rule

%----------------------------------------------------------------------------------------
%	DOCUMENT STRUCTURE COMMANDS
%	Skip this unless you know what you're doing
%----------------------------------------------------------------------------------------

% Header and footer for when a page split occurs within a problem environment
\newcommand{\enterProblemHeader}[1]{
\nobreak\extramarks{#1}{#1 continued on next page\ldots}\nobreak
\nobreak\extramarks{#1 (continued)}{#1 continued on next page\ldots}\nobreak
}

% Header and footer for when a page split occurs between problem environments
\newcommand{\exitProblemHeader}[1]{
\nobreak\extramarks{#1 (continued)}{#1 continued on next page\ldots}\nobreak
\nobreak\extramarks{#1}{}\nobreak
}

\setcounter{secnumdepth}{0} % Removes default section numbers
\newcounter{homeworkProblemCounter} % Creates a counter to keep track of the number of problems

\newcommand{\homeworkProblemName}{}
\newenvironment{homeworkProblem}[1][Problem \arabic{homeworkProblemCounter}]{ % Makes a new environment called homeworkProblem which takes 1 argument (custom name) but the default is "Problem #"
\stepcounter{homeworkProblemCounter} % Increase counter for number of problems
\renewcommand{\homeworkProblemName}{#1} % Assign \homeworkProblemName the name of the problem
\section{\homeworkProblemName} % Make a section in the document with the custom problem count
\enterProblemHeader{\homeworkProblemName} % Header and footer within the environment
}{
\exitProblemHeader{\homeworkProblemName} % Header and footer after the environment
}

\newcommand{\problemAnswer}[1]{ % Defines the problem answer command with the content as the only argument
\noindent\framebox[\columnwidth][c]{\begin{minipage}{0.98\columnwidth}#1\end{minipage}} % Makes the box around the problem answer and puts the content inside
}

\newcommand{\homeworkSectionName}{}
\newenvironment{homeworkSection}[1]{ % New environment for sections within homework problems, takes 1 argument - the name of the section
\renewcommand{\homeworkSectionName}{#1} % Assign \homeworkSectionName to the name of the section from the environment argument
\subsection{\homeworkSectionName} % Make a subsection with the custom name of the subsection
\enterProblemHeader{\homeworkProblemName\ [\homeworkSectionName]} % Header and footer within the environment
}{
\enterProblemHeader{\homeworkProblemName} % Header and footer after the environment
}

%----------------------------------------------------------------------------------------
%   DOCUMENT DEFINITIONS  	
%  
%----------------------------------------------------------------------------------------

\newcommand{\hmwkTitle}{Homework 10} % Assignment title
\newcommand{\hmwkClass}{APKC} % Course/class
\newcommand{\hmwkClassTime}{} % Class/lecture time
\newcommand{\hmwkClassInstructor}{} % Teacher/lecturer
\newcommand{\hmwkAuthorName}{Sebastien Duc} % Your name
%----------------------------------------------------------------------------------------
%   MY INCLUDES	
%   my package include and commands	
%----------------------------------------------------------------------------------------

\usepackage{amsmath}
\usepackage{amssymb}
\usepackage{amsthm}

\newtheorem{thm}{Theorem}
\newtheorem{lemma}{Lemma}

\usepackage{algorithm,algorithmicx,algpseudocode}
\usepackage{hyperref}

\title{APKC - Homework 10}
\author{Sebastien Duc}
\date{\today}


\newcommand {\zpz}[1]{\mathbb{Z}/#1\mathbb{Z}}
\newcommand {\polyf}{\mathbb{Z}[x]}
\newcommand {\sign}{\mathrm{sign}}
\newcommand {\thus}{\Rightarrow\:}
\newcommand {\nequiv}{\not\equiv}
\newcommand {\fact}{\mathrm{fact}}
\newcommand {\ord}{\mathrm{ord}}
\newcommand {\lcm}{\mathrm{lcm}}
\newcommand {\Lfunc}[2]{L_x[#1;#2]}
\newcommand {\Ls}{\Lfunc{s}{\beta}}
\newcommand {\Lr}{\Lfunc{r}{\alpha}}
\newcommand {\Lrs}{\Lfunc{r-s}{-\alpha(r-s)/\beta}}

\begin{document}

\section{10.1}
First we need to find all primes smaller than $B$. For that we can use a sieve. Let $C_{SIEVE}(B)$ be the cost for a sieve which finds prime smaller than $B$.
Since we need to find the prime power smaller than $B$, for every prime we found we need at most $\log(B)$ multiplications, because $2$ is the smallest prime 
and $2^{\log B} = B$. We know that for a given $B$  we find $\pi(B)$ primes smaller than $B$. Thus we get a complexity in $O(C_{SIEVE} + \pi(B)\log(B)M(B))$, where
$M(B)$ is the cost of a multiplication when the two operands are $\leq B$. 

After that we need to compute $a^M - 1$ so that we can compute $\gcd(n,a^M -1)$.
But before we will prove that we can compute $a^M-1 \bmod n$ instead, because 
\[
    \gcd(n,a^M-1) = \gcd(n,a^M -1 \bmod n).
\]
Suppose $d = \gcd(n,a^M-1)$ then $n = kd$ and $a^M - 1 = ld$ for $k,l\in \mathbb{Z}$. Thus $a^M - 1 \bmod n = ld \bmod n = ld - rkd = (l-rk)l$,
and $\gcd(k,l-rk) = \gcd(k,l) = 1$. Hence $\gcd(n,a^M-1) = \gcd(n,a^M -1 \bmod n)$.

To compute $a^M -1 \bmod n$ we can use square and multiply. Furthermore we do the following $r = 1$ and for every prime power $p^k$ we found we compute 
$r = r \cdot a^{p^k} \bmod n$. So in the end get $r =  a^M$, we just compute $r = r-1 \bmod n = a^M -1 \bmod n$.
For every prime power (we have $\pi(B)$ of them) we need $\log(p^k) \leq \log(B)$ multiplications.
Thus the cost for to get $a^M -1 \bmod n$ is in $O(\log(B)\pi(B)M(n))$ where $M(n)$ is the cost of a multiplication modulo $n$.

Finally we can compute the gcd in $O(\log^2 n)$ using binary gcd algorithm.
Therefore the total complexity is
\[
    O(C_{SIEVE} + \pi(B)\log(B)M(B) + \log(B)\pi(B)M(n) + \log^2 n).
\]
We know that $C_{SIEVE} = B\log\log B$ when using the sieve of Eratosthenes. We also know that $\pi(B) \approx \frac{B}{\log B}$.
And $M(n)$ is in $O(\log^2 n )$ when using schoolbook multiplication. Similarly $M(B)$ is in $O(\log^2 B)$.
Consequently the total complexity is
\[
    O(B\log^2n).
\]
\section{10.2}
The probability of having a $\Lfunc{s}{\beta}$-smooth integer $n$ given that $n \leq \Lfunc{r}{\alpha}$ is given by
\[
    \frac{\psi(\Lfunc{r}{\alpha},\Lfunc{s}{\beta})}{\Lfunc{r}{\alpha}}.
\]
Let us see how to compute $\psi(\Lr,\Ls)$.
We know that for an arbitrary $\epsilon>0$ and for $u \leq (\log\Lr)^{1-\epsilon}$, $\psi(\Lr,\Lr^{1/u}) = \Lr u^{-u+o(1)}$.
Therefore we want to have $u\leq (\log\Lr)^{1-\epsilon}$ s.t. $\Lr^{1/u} = \Ls$.
We have
\[
    \begin{array}{l r l}
        & \Lr^{1/u} &= \Ls       \\
   \iff & \log\Lr u^{-1} &= \log\Ls\\
   \iff & (\alpha + o(1))(\log x)^r(\log\log x)^{1-r} u^{-1} &= (\beta + o(1))(\log x)^s (\log\log x)^{1-s}\\
   \iff & u &= \frac{\alpha + o(1)}{\beta + o(1)} (\log x)^{r-s}(\log\log x)^{s-r}\\
    \end{array}
\]
We still need to show that for some $\epsilon > 0$, our $u$ is smaller or equal to $(\log\Lr)^{1-\epsilon}$.
Indeed, \[(\log x)^{r-s} \leq (\log x)^{r(1-\epsilon)}\] for $0 < \epsilon < s/r$, \[(\log\log x)^{s-r} \leq (\log\log x)^{(1-r)(1-\epsilon)}\] for $0 < \epsilon < \frac{1-s}{1-r}$
and finally \[\frac{\alpha + o(1)}{\beta + o(1)} \leq (\alpha + o(1))^{1-\epsilon}\] for $\alpha^{\epsilon} \leq \beta + o(1) \:\thus 0 < \epsilon \leq \frac{\log\beta}{\log\alpha}$ if 
$\alpha \geq 1$ and $\epsilon > \max\{0,\frac{\log\beta}{\log\alpha}\}$ if $\alpha < 1$.

Given our $u$, we get
\[
    \begin{split}
        \psi(\Lr,\Ls) &= \Lr e^{-u\log u + o(1)}\\
                      &= \Lr e^{-(\frac{\alpha}{\beta} + o(1))(\log x)^{r-s}(\log\log x)^{s-r}\log u}
    \end{split}
\]
Therefore the probability is given by
\[
    \begin{split}
        &\exp\left\{-\left(\frac{\alpha}{\beta} + o(1)\right)(\log x)^{r-s}(\log\log x)^{r-s}\log u\right\}\\
    \mbox{with } &\log u = \left(\log\left(\frac{\alpha}{\beta} + o(1)\right) + (r-s)\log\log x + (s-r)\log\log\log x \right)\\
           \thus &\mbox{ the probability is } \Lrs^{1 + \frac{\log(\frac{\alpha}{\beta} + o(1)) + (s-r)\log\log\log x}{\log\log x}}
    \end{split}
\]
But the probability tends to $\Lrs$ as $x$ goes to infinity since 
\[
    \lim_{x\rightarrow\infty}\frac{\log(\frac{\alpha}{\beta} + o(1)) + (s-r)\log\log\log x}{\log\log x} = 0.
\]
Hence the result.
\section{10.3}
\subsection{10.3.1}
Let $f(X) = \sum_{i=0}^n a_i X^i \in \mathbb{Z}[X]$.
$q$ divides $f(p)$ iff $f(r)\mid f(p)$. Furthermore
\[
    \begin{split}
        f(p) &= f(r + kf(r)) = \sum_{i=0}^n a_i(r+kf(r))^i\\
             &= \sum_{i=0}^n a_i \sum_{j=0}^i \binom{i}{j}r^j (kf(r))^{i-j}\\
             &= \sum_{i=0}^n a_i \sum_{j=0}^{i-1} \binom{i}{j}r^j (kf(r))^{i-j} + \sum_{i=0}^n a_i r^i\\
             &= f(r)\sum_{i=0}^n a_i \sum_{j=0}^{i-1} \binom{i}{j}r^j k^{i-j}(f(r))^{i-1-j} + f(r)
    \end{split}
\]
Consequently $f(r) \mid f(p)$.
\subsection{10.3.2}
For this exercise we use the hint we just proved combined with cyclotomic polynomials. 
We want to find two primes $p,q$ such that $q \mid p^k - 1$ but $q \nmid p^s - 1$ for all $s\mid k$.
We know that the $k$-th cyclotomic polynomial $\Phi_k(X)$ divides $X^k - 1$ and but does not divide $X^s - 1$ for all $s \mid k$.

Therefore what will do is, first find $r$ s.t. $q =\Phi_k(r)$ is prime then using the hint we try to find $p = r + a\cdot q$ prime.
We just have to make sure that $p,q$ are in the correct range. The property of cyclotomic polynomials ensures that $p,q$ have the required properties.

To find the generator for $k=1$, we use the fact the for a $h \in (\mathbb{F}_p)^*$, if $g = h^{(p-1)/q} \neq 1$ then $g$ has order $q$.

\textbf{Remark:} the case $k=7$, $(L,B) = (160,1024)$ is a bit different as $p < q$ if $p^7$ has 1024 bits. Therefore we are forced to choose $p$ such that $p^7$
 has more than 1024 bits.

\end{document}
